\documentclass{beamer}
\usetheme{Madrid}
\usecolortheme{default}

\usepackage[utf8]{inputenc}
\usepackage[T1]{fontenc}
\usepackage[english]{babel}
\usepackage{graphicx}
\usepackage{booktabs}
\usepackage{amsmath}

% Show a title slide at the start of each section
\AtBeginSection[]{
    \begin{frame}
        \centering
        \vfill
        {\huge \insertsectionhead}
        \vfill
    \end{frame}
}

\title{Green Fine Tuning for Molecular Property Prediction}
\author{Emanuele Fontana}
\date{}

\begin{document}

\begin{frame}
    \titlepage
\end{frame}

\begin{frame}{Outline}
    \tableofcontents
\end{frame}
\section{Business Understanding}

\begin{frame}{Context and Motivation}
    \begin{itemize}
        \item \textbf{Context}: Rise of Deep Learning in Drug Discovery (Transformers, GNNs).
        \item \textbf{Problem}: "Red AI" trend leading to high computational costs and energy consumption.
        \item \textbf{Solution}: \textbf{Green AI} paradigm - sustainable AI.
        \item \textbf{Alignment}: UN 2030 Agenda for Sustainable Development.
        \begin{itemize}
            \item Goal 12: Responsible Consumption and Production.
            \item Goal 13: Climate Action.
        \end{itemize}
    \end{itemize}
\end{frame}

\begin{frame}{Business Objectives}
    \begin{block}{Primary Goal}
        Demonstrate that sustainable practices can be integrated into molecular property prediction pipelines without significant loss in performance.
    \end{block}
    \begin{itemize}
        \item \textbf{Sustainability}: Drastically reduce CO$_2$eq emissions during fine-tuning.
        \item \textbf{Reliability}: Maintain predictive accuracy for real-world use.
    \end{itemize}
    \begin{block}{Expectations}
        \begin{itemize}
            \item CO$_2$ reduction: 20\% - 50\%.
            \item Performance margin: Within 10\% of traditional methods.
        \end{itemize}
    \end{block}
\end{frame}


\section{Data Understanding}


\begin{frame}{Datasets (MoleculeNet)}
    \textbf{Classification Tasks}:
    \begin{itemize}
        \item \textbf{HIV}: Predict if HIV is active or inactive.
        \item \textbf{BACE}: Predict inhibitors of BACE-1 enzyme (Alzheimer's).
        \item \textbf{BBBP}: Predict blood-brain barrier penetration.
    \end{itemize}
    
    \vspace{0.5cm}
    \textbf{Regression Tasks}:
    \begin{itemize}
        \item \textbf{Lipophilicity}: Predict octanol/water distribution coefficient.
        \item \textbf{Malaria}: Predict inhibition of malaria parasite growth.
        \item \textbf{CEP}: Predict efficiency of organic photovoltaic molecules.
    \end{itemize}
\end{frame}

\begin{frame}{Dataset Details}
    \begin{table}
        \centering
        \begin{tabular}{lllr}
            \toprule
            \textbf{Dataset} & \textbf{Task} & \textbf{Key Components} & \textbf{Rows} \\
            \midrule
            BACE & Classification & mol, CID, Class, pIC50 & 1{,}513 \\
            BBBP & Classification & num, name, p\_np, smiles & 2{,}050 \\
            HIV & Classification & smiles, activity & 41{,}126 \\
            CEP & Regression & smiles, PCE & 29{,}978 \\
            Malaria & Regression & smiles, activity & 9{,}999 \\
            Lipophilicity & Regression & CMPD\_CHEMBLID, exp, smiles & 4{,}200 \\
            \bottomrule
        \end{tabular}
    \end{table}
\end{frame}

\begin{frame}{Data Representation: SMILES}
    \textbf{SMILES} (Simplified Molecular Input Line Entry System):
    \begin{itemize}
        \item ASCII strings representing chemical structures.
        \item \textbf{Atoms}: C, N, O, etc. (Upper: aliphatic, Lower: aromatic).
        \item \textbf{Bonds}: Single (implicit), Double (=), Triple (\#).
        \item \textbf{Branching}: Parentheses ().
        \item \textbf{Rings}: Numbers (e.g., C1CCCCC1).
    \end{itemize}
    \vspace{0.5cm}
    Crucial for converting 3D structures into 1D sequences for Transformers or Graphs for GNNs.
\end{frame}

\section{Data Preparation}
\begin{frame}{Preprocessing Pipeline}
    \begin{enumerate}
        \item \textbf{SMILES Validation}:
        \begin{itemize}
            \item \textbf{Parsing}: Check syntax errors using RDKit.
            \item \textbf{Sanitization}: Valence checks, Kekulization, Aromaticity detection.
        \end{itemize}
        \item \textbf{Data Variants}:
        \begin{itemize}
            \item \textbf{Transformer Variant}: SMILES string + Target label.
            \item \textbf{Graph Variant}: Molecular graphs generated from SMILES.
        \end{itemize}
    \end{enumerate}
\end{frame}

\begin{frame}{Graph Generation (for GNNs)}
    Conversion of SMILES to Graph $G = (V, E)$ using RDKit.
    \begin{columns}
        \column{0.5\textwidth}
        \textbf{Node Features ($V$)}:
        \begin{itemize}
            \item Atomic Number
            \item Chirality
            \item Degree
            \item Formal Charge
            \item Hybridization
            \item Aromaticity
        \end{itemize}
        
        \column{0.5\textwidth}
        \textbf{Edge Features ($E$)}:
        \begin{itemize}
            \item Bond Type
            \item Stereochemistry
            \item Conjugation
        \end{itemize}
    \end{columns}
    \vspace{0.5cm}
    Serialized as PyTorch Geometric \texttt{Data} objects (.pt files).
\end{frame}

\section{Modeling}
\begin{frame}{Architectures}
    \begin{columns}
        \column{0.5\textwidth}
        \textbf{Sequence-based (Transformers)}
        \begin{itemize}
            \item Input: SMILES strings.
            \item Models:
            \begin{itemize}
                \item ChemBERTa
                \item ChemBERTa-2
                \item SELFormer
                \item SMILES-BERT
            \end{itemize}
        \end{itemize}
        
        \column{0.5\textwidth}
        \textbf{Graph-based (GNNs)}
        \begin{itemize}
            \item Input: Molecular Graphs.
            \item Model:
            \begin{itemize}
                \item \textbf{GraphMAE}: Masked Autoencoder approach for self-supervised learning.
            \end{itemize}
        \end{itemize}
    \end{columns}
\end{frame}

\begin{frame}{Green FineTuning (GFT) Strategy}
    \textbf{Core Idea}: Stop training when marginal performance gain doesn't justify energy cost.
    
    \begin{block}{Instantaneous GFT Ratio}
        \vspace{-0.1cm}
        \begin{equation*}
            GFT_t = \frac{\Delta P_t}{\Delta E_t} = \frac{Perf_t - Perf_{t-1}}{Emission_t - Emission_{t-1}}
        \end{equation*}
        \vspace{-0.2cm}
        where:
        \begin{itemize}
            \item $\Delta P_t$ = improvement in validation metric at epoch $t$
            \item $\Delta E_t$ = incremental CO$_2$eq emitted during epoch $t$
        \end{itemize}
    \end{block}
    
    \vspace{-0.1cm}
    \begin{block}{Smoothed Trend (EMA)}
        \vspace{-0.1cm}
        To stabilize the volatile instantaneous efficiency:
        \begin{equation*}
            S_t = \alpha\,GFT_t + (1-\alpha)\,S_{t-1}
        \end{equation*}
        \vspace{-0.3cm}
        where $\alpha=0.9$. Initialization: $S_0 = GFT_0$ (or $0$).
        \vspace{0.1cm}
    \end{block}
\end{frame}

\begin{frame}{GFT Stopping Rule}
    \begin{block}{Stopping Condition}
        Training stops when instantaneous efficiency falls below the smoothed trend:
        \begin{equation*}
            \text{Stop if } GFT_t < \beta \cdot S_t
        \end{equation*}
        where $\beta=0.2$ is the tolerance factor.
    \end{block}

    \begin{block}{Implementation Details}
        \begin{itemize}
            \item $Perf_t$ is the validation metric:
            \begin{itemize}
                \item ROC-AUC for classification tasks
                \item Negative RMSE for regression tasks (to maintain "higher is better")
            \end{itemize}
            \item Warmup period: 3 epochs (Transformers), variable for GraphMAE
            \item Energy tracking: CodeCarbon library
        \end{itemize}
    \end{block}
\end{frame}

\section{Evaluation}
\begin{frame}{Results: Classification Tasks}
    \begin{itemize}
        \item \textbf{BACE}: 
        \begin{itemize}
            \item Transformers: +1.6\% to +7.0\% performance, -22\% to -60\% emissions.
            \item Early stopping prevents overfitting.
        \end{itemize}
        \item \textbf{HIV}:
        \begin{itemize}
            \item Strong improvements (e.g., SELFormer +16.2\% perf, -44\% emissions).
            \item ChemBERTa-2: Stable perf, -60.6\% emissions.
        \end{itemize}
        \item \textbf{BBBP}:
        \begin{itemize}
            \item Mixed results. Some models degrade (ChemBERTa -11.8\%).
            \item SELFormer resilient (+6.9\% perf).
        \end{itemize}
    \end{itemize}
\end{frame}

\begin{frame}{Trade-off: BACE}
    \begin{center}
        \includegraphics[width=0.9\textwidth]{../Documentazione/img/tradeoff_experiments_bace.png}
    \end{center}
\end{frame}

\begin{frame}{Trade-off: HIV}
    \begin{center}
        \includegraphics[width=0.9\textwidth]{../Documentazione/img/tradeoff_experiments_hiv.png}
    \end{center}
\end{frame}

\begin{frame}{Trade-off: BBBP}
    \begin{center}
        \includegraphics[width=0.9\textwidth]{../Documentazione/img/tradeoff_experiments_bbbp.png}
    \end{center}
\end{frame}

\begin{frame}{Results: Regression Tasks}
    \begin{itemize}
        \item \textbf{CEP}:
        \begin{itemize}
            \item Excellent results. SELFormer: +50.6\% perf, -59.7\% emissions.
            \item GraphMAE: Trade-off between warmup and performance.
        \end{itemize}
        \item \textbf{Lipophilicity}:
        \begin{itemize}
            \item Most challenging. Many models degrade.
            \item Requires extended fine-tuning.
        \end{itemize}
        \item \textbf{Malaria}:
        \begin{itemize}
            \item Moderate success. ChemBERTa-2: +3.3\% perf, -23.6\% emissions.
            \item GraphMAE benefits from longer warmup.
        \end{itemize}
    \end{itemize}
\end{frame}

\begin{frame}{Trade-off: CEP}
    \begin{center}
        \includegraphics[width=0.9\textwidth]{../Documentazione/img/tradeoff_experiments_cep.png}
    \end{center}
\end{frame}

\begin{frame}{Trade-off: Lipophilicity}
    \begin{center}
        \includegraphics[width=0.9\textwidth]{../Documentazione/img/tradeoff_experiments_lipophilicity.png}
    \end{center}
\end{frame}

\begin{frame}{Trade-off: Malaria}
    \begin{center}
        \includegraphics[width=0.9\textwidth]{../Documentazione/img/tradeoff_experiments_malaria.png}
    \end{center}
\end{frame}

\begin{frame}{Explainability Analysis}
    \begin{itemize}
        \item \textbf{Motivation}: understand which factors drive emissions and performance to improve GFT.
        \item Use \textbf{Partial Correlation} to isolate each feature's unique contribution while controlling for others.
        \item Valuable when features are collinear across experiments.
    \end{itemize}
\end{frame}


\begin{frame}{Factors Influencing Emissions}
    \begin{itemize}
        \item \textbf{Top drivers}: Dataset Size ($\rho=0.65$), Epoch Efficiency ($\rho=0.26$).
        \item \textbf{Interactions}: Epochs Used shows interaction effects with dataset size and efficiency.
        \item \textbf{Model architecture}: moderate influence (num\_layers, hidden\_size, $\rho\approx0.12$).
        \item \textbf{Early stopping}: minimal direct effect ($\rho=-0.05$) — its benefit is mediated by fewer epochs used.
    \end{itemize}
    \begin{center}
        \includegraphics[width=0.8\textwidth]{../Documentazione/img/partial_correlation_importance.png}
    \end{center}
\end{frame}



\section{Conclusions}
\begin{frame}{Key Findings}
    \begin{itemize}
        \item \textbf{GFT Effectiveness}: Drastic emission reductions (60\%-80\%) often possible with maintained or improved accuracy.
        \item \textbf{Classification vs Regression}:
        \begin{itemize}
            \item Classification: Often "win-win" (prevents overfitting).
            \item Regression: Clearer trade-off, requires careful tuning.
        \end{itemize}
        \item \textbf{Explainability Insights}:
        \begin{itemize}
            \item \textbf{Dataset Size}: Dominant factor for emissions.
            \item \textbf{Early Stopping}: Positively correlates with performance.
        \end{itemize}
    \end{itemize}
\end{frame}

\begin{frame}{Recommendations}
    \begin{enumerate}
        \item \textbf{Implement GFT}: Standardize adaptive early stopping.
        \item \textbf{Optimize Dataset Usage}: Focus on data quality over quantity.
        \item \textbf{Task-Specific Tuning}: Adjust warmup epochs based on task complexity (especially for Regression/GNNs).
        \item \textbf{Model Selection}: Transformers show high resilience in classification tasks.
    \end{enumerate}
    
    \vspace{1cm}
    \centering
    \textbf{Thank You!}
\end{frame}

\end{document}
